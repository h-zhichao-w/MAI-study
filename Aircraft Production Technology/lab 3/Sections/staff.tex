\section{Staff} \label{sec:Staff}

\subsection{Workers}
Assuming an average workspace of 1.5 $\mathrm{m^2}$ per worker, it can be estimated that the maximum number of workers that can be working at any given time is 16 (25/1.5). Assuming three shifts of work (morning, afternoon, and night), it can be estimated that the total number of workers in a day is 48 (16 x 3). However, the practical number of workers would be less than this, as the number of products needed at this time is only 100.

\subsection{Employees}
Employees are needed to manage the factory and the workers. The number of employees needed is estimated to be 2, including the safety and technical manager and the accountant.

\subsection{Guards}
Guards are needed to ensure the safety of the factory. The number of guards needed is estimated to be 6, including the three shifts (2 for one shift).

\subsection{Estimated time for production }
Assuming 16 workers work together at a time and are separated into the assembly line and the quality check, as each worker has a specific task to complete, the assembly line sequence suggests that each station would take around 5-10 minutes to complete. Therefore, the total assembly time for one seat would be approximately 25-50 minutes. Assuming the factory operates 24 hours a day, the production is 28. To completely produce a batch of 100 pieces, it would take 4 days.

For quality checks, if we assume that each check takes an average of 5 minutes per item, then the quality control process would take approximately 45 minutes. Therefore, assuming the factory operates 24 hours a day, the number of products that can undergo quality checks and wait for delivery is 32. 

Overall, the factory would need to operate for 4 days to produce 100 seats.